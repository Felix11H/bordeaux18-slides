
\documentclass[usenames,dvipsnames,11pt]{beamer}

% contains the complete preamble
\usepackage{present}
\usepackage{custom_underline}
         		               		
\title{}
\author[Felix Z.~Hoffmann]{Felix Z.~Hoffmann}
\institute{}
\date{\today}     


\begin{document}


\begin{frame}
  %
  \vspace{0.6cm}
  
  \begin{center} 
    \Large Open \& reproducible research - What can we do in practice?
  \end{center}
  
  \vspace{0.8cm}

  \small
  \begin{columns}[t]
    %%%%
    \begin{column}{0.4\textwidth}        
      \textbf{Presented by}
      
      %\vspace{0.18cm}
      \begin{itemize}[leftmargin=0.6cm]
        \itemsep0pt
      \item[] Felix Z.~Hoffmann
        \item[] @Felix11H
        \item[] \href{http://felix11h.github.io/}{felix11h.github.io/}
      \end{itemize}

      \vspace{0.28cm}
      \textbf{Slides}
      
      %\vspace{0.18cm}
      \begin{itemize}[leftmargin=0.6cm]
        %% \item[] Slideshare:\\ %
        %%   \href{}{git.io/vFEbX}
        \item[] GitHub: %
          \href{http://bit.ly/bx18s}{bit.ly/bx18s}
      \end{itemize}
    \end{column}
    %%%% 
    \begin{column}{0.59\textwidth}
      \textbf{Resources and links}
      \vspace{-0.1cm}
       
      \begin{itemize}[leftmargin=0.6cm]
        \itemsep4pt
        \item[] Open Science Fellows Program: %
          \href{http://bit.ly/osfprog}{bit.ly/osfprog}
        \item[] project description: %
          \href{http://bit.ly/osfproj}{bit.ly/osfproj}
        \item[] prototype: \href{http://bit.ly/osrep}{bit.ly/osrep}
      \end{itemize}



    \end{column}
    %%%%
  \end{columns}

  \vspace{0.5cm}

  %% \begin{center}
  %%   \includegraphics[width=1.7cm]{ccby40.png}

  %%   \vspace{0.1cm}

  %%   \scriptsize This work is licensed under a
  %%   \href{https://creativecommons.org/licenses/by/4.0/}{Creative
  %%     Commons Attribution 4.0 International License.}
  %% \end{center}

  %% \begin{center}
  %%   \includegraphics[width=\textwidth]{img/gsoc14_banner.png}
  %% \end{center}

  \begin{figure}
  \centering
  \includegraphics[width=\textwidth]{%
  img/wosf_header.jpg} %
\end{figure}



\end{frame}



% pre-Intro: About me - what qualifies me to tell you
% anything about reproducibility/open science.
% Who am I to talk?

\begin{frame}{}

  - Research Lab

  - Google Summer of Code

  - Wikimedia Open Science Fellow

   
  
\end{frame}


% I'm not the only caring about reproducibility...
\begin{frame}{The reproducibility crisis}

  \begin{figure}
    \centering
    \includegraphics[width=0.85\textwidth]{%
    img/reproducibility_crisis_q.jpeg} %
  \end{figure}
  

  
  
\end{frame}


% Computational reproducibility shouldn't be hard - right?
\begin{frame}{Computational reproducibility should be easy...}

  % 
  \begin{columns}
    %
    \begin{column}{.5\textwidth}
      \begin{figure}
        \centering
        \includegraphics[width=0.85\textwidth]{%
          monkey.png}  %
      \end{figure}
      
    \end{column}
    %
    \begin{column}{.5\textwidth}

      \begin{figure}
        \centering
        \includegraphics[width=0.85\textwidth]{%
          computer.png}  %
      \end{figure}
      
    \end{column}
  \end{columns}

  \begin{columns}
    %
    \begin{column}{.5\textwidth}
      \begin{center}
        Hard
      \end{center}
      
    \end{column}
    %
    \begin{column}{.5\textwidth}
      \begin{center}
        Easy?
      \end{center}
      
    \end{column}
  \end{columns}
  
\end{frame}


\begin{frame}{Computational reproducibility should be easy...}

  \vspace{0.7cm}
  
  \begin{itemize}[leftmargin=1.2cm, rightmargin=1cm]

    \itemsep10pt

  \item[1.] cheap and universal access to computers (as opposed to a lab)
  \item[2.] running code is inexpensive and unproblematic (compared to replicating an experiment)
  \item[3.] can easily share code \& data that allow direction reproduction
    
    
  \end{itemize}

  % 
  \begin{columns}
    %
    \begin{column}{.625\textwidth}
      \minipage[c][0.45\textheight][s]{\columnwidth}

      \vspace{1.2cm}
      \begin{center}
        \textit{Is there a reproducibility crisis\\ in computational research?}
      \end{center}        
      
      \endminipage      
    \end{column}
    %
    \begin{column}{.375\textwidth}
      \onslide<1->
      \vspace{-1cm}
      \begin{figure}
        \centering
        \includegraphics[width=\textwidth]{%
          img/computer.png} %
      \end{figure}
      
    \end{column}
  \end{columns}
  

  \pnote{
    
    But as many of you have probably experienced,\\
    code is often not shared.\\

    But surely, journals implementing policies \\
    should help?

    Well, let's take a look:
    
  }

  
\end{frame}



% In experience, researchers don't like to share code.
% So how well does it work when researchers are
% obligated to share?

\begin{frame}{Sharing of code \& data mandatory in many journals}

  
  \begin{figure}
    \centering
    \includegraphics<1>[width=0.95\textwidth]{%
      img/science_policy.png} %
    \includegraphics<2>[width=0.95\textwidth]{%
      img/science_policy_t1.png} %
    \includegraphics<3>[width=0.95\textwidth]{%
      img/science_policy_t2.png} %
  \end{figure}

  \vspace{0.4cm}
  
  \begin{flushright}
    \small Policy of \textit{Science} since February 11, 2011
  \end{flushright}

  \pnote{
    
    Now that we established computational reproducibility\\
    should be simple and journals are demanding\\
    sharing anyway, we're good, right?

    How effect is such an policy?
    
  }
  
\end{frame}

\begin{frame}{}
  
  \begin{figure}
    \centering
    \includegraphics[width=0.875\textwidth]{%
      img/stodden2018_cover3.png} %
  \end{figure}
  
  \vspace{0.015cm}

  \begin{center}
    Out of 206 computational studies in \textit{Science} since 2011, \\
    26 provided code \& data directly  
  \end{center}
  

  \source{\cite{Stodden2018}}

  \pnote{
    
    To the remaining 180 studies \\
    the authors sent Emails asking \\
    for the code for the studies
    
  }

  
\end{frame}


\begin{frame}{A few responses...}
  
  \begin{figure}
    \centering
    \includegraphics<1>[width=0.95\textwidth]{%
      img/stodden2018_re1.png} %
    \includegraphics<2->[width=0.95\textwidth]{%
      img/stodden2018_re1_op.png} %

  \end{figure}

  \begin{figure}
    \centering
    \includegraphics<1>[width=0.95\textwidth]{%
      img/stodden2018_re2_op.png} %
    \includegraphics<2>[width=0.95\textwidth]{%
      img/stodden2018_re2.png} %
    \includegraphics<3>[width=0.95\textwidth]{%
      img/stodden2018_re2_op.png} %

  \end{figure}

  \begin{figure}
    \centering
    \includegraphics<1-2>[width=0.95\textwidth]{%
      img/stodden2018_re3_op.png} %
    \includegraphics<3>[width=0.95\textwidth]{%
      img/stodden2018_re3.png} %
  \end{figure}

  \source{\cite{Stodden2018}}
  
\end{frame}



\begin{frame}{\large   Of $N=206$ articles published in \textit{Science} since 2011...}


  \begin{figure}
    \centering
    \includegraphics[width=0.9\textwidth]{%
      img/stodden2018_table1.png} %
  \end{figure}

  \vspace{0.1cm}
  
  \begin{center}
    $\Rightarrow$ Code \& data could be retrieved for 91 out of 206 studies.    
  \end{center}

  \source{\cite{Stodden2018}}
  
\end{frame}




% but in Neuroscience very positive movement!
\begin{frame}{Open Source for Neuroscience}

  \begin{figure}
    \centering
    \includegraphics<1>[width=0.9\textwidth]{%
      img/open_source_neuroscience.png} %
    \includegraphics<2>[width=\textwidth]{%
      img/pledge_2.png} %
    \includegraphics<3>[width=\textwidth]{%
      img/pledge_underline_2.png} %
  \end{figure}

  \onslide<2->
  \begin{center}
    \href{http://opensourceforneuroscience.org/}{opensourceforneuroscience.org/}
  \end{center}

  \pnote{
    
    As of 2018/04: 200 neuroscientist have made this pledge
    
  }
  
\end{frame}


% It's not great, but especially in neuroscience there
% is a positive trend. With more and more code being
% available soon, computational reproducibility
% shouldn't be a problem, yes?



\begin{frame}{Reproducibility when code was available}

  \vspace{0.05cm}
  
  Tried to reproduce randomly selected 22 studies (out of 56 that were judged as potentially reproducible)

  \vfill
  
  \onslide<2->
  \begin{tabular}{|p{0.9\textwidth}}
    Even when code was available, more than half of studies were reproducible only with \textit{\textcolor{red}{significant effort!}}  
  \end{tabular}	     

  \vfill

  \vspace{0.2cm}

  \onslide<3-6>{

    Problems:
  \vspace{-0.15cm}

  \begin{columns}[T]
    %
    \begin{column}{.5\textwidth}
      %% \minipage[c][0.3\textheight][s]{\columnwidth}
      \begin{itemize}[leftmargin=0.75cm]

      \item<3->[-] impossible to reproduce (missing code, data or methodology)
      \item<4->[-] required tedious effort (e.g.\ download large number of individual data sets)
        
        
      \end{itemize}
      %% \endminipage      

    \end{column}
    %
    \begin{column}{.5\textwidth}
      %% \minipage[c][0.3\textheight][s]{\columnwidth}
      
      \begin{itemize}[leftmargin=*]
      \item<5->[-] required intellectual effort (e.g.\ knowledge of past articles, implementing given pseudo code)
      \item<6->[-] required tweaking (e.g.\ mising parameters, minor methods steps)
            
      \end{itemize}
      %% \endminipage
      
    \end{column}
    %
  \end{columns}}

  \source{\cite{Stodden2018}}
    
\end{frame}



\begin{frame}{Reproducibility when code was available}

    \begin{figure}
      \centering
      \includegraphics[width=0.85\textwidth]{%
      img/stodden2018_table4_3-3.png} %
    \end{figure}

    \vspace{-1.5cm}
  
\begin{center}
  {\large Computational reproducibility remains difficult \\even
    when code is available!}
      
\end{center}
  
    
\end{frame}


\begin{frame}{}
  
  \begin{figure}
    \centering
    \includegraphics[width=0.95\textwidth]{%
    img/matlab.png} %
  \end{figure}

  \begin{center}
  \large How to publish our research so that (computational) results are reproducible?    
  \end{center}

  \pnote{
    
    Achieving computational reproducibility is on us,\\
    all of are writing code - even if it's \\
    just analysis in MATLAB.

    It's a challenge to do things differently from \\
    how they were done until now.
    
  }
  
\end{frame}



% If I'm talking the talk, I have to walk the walk.
% How can I achieve reproducibility in my research
% project I'm trying to publish?

\begin{frame}{A complex computational study - Reproducible?}

  Multi. Generation of large networks. Inherently computational - only generated data
  
  
  
\end{frame}


% Wikimedia fellowship helped!
\begin{frame}{Open Science Fellowship}

\begin{figure}
  \centering
  \includegraphics[width=0.9\textwidth]{%
    img/logo_wosf.png} %
\end{figure}

  
\end{frame}




\begin{frame}{Open Science Fellowship}

  % 
  \begin{columns}
    %
    \begin{column}{.55\textwidth}


      \begin{figure}
        \centering
        \includegraphics[width=0.96\textwidth]{%
        img/logo_wosf_full.png} %
      \end{figure}
      
     

    \end{column}
    %
    \begin{column}{.45\textwidth}
      \minipage[c][0.625\textheight][s]{\columnwidth}

      \vspace{0.2cm}
      
      \begin{itemize}[leftmargin=*]

      \item[-] runs from October to June
        
      \item[-] fellows are paired with a mentor who supports the progress

      \item[-] financial support provided

      \item[-] training seminars \& workshops
        
      \item[-] program in German
        
      \end{itemize}

      \endminipage      
    \end{column}
  \end{columns}
  
\end{frame}




\begin{frame}{Open Science Fellowship}

  %% picture of the current fellows
  \begin{figure}
    \centering
    \includegraphics[width=0.875\textwidth]{%
    img/WOSF2017.jpg} %
  \end{figure}

  \onslide<2->
  \begin{center}
  application for 3rd round open: \href{http://bit.ly/osfprog}{bit.ly/osfprog}
  \end{center}
  
  \source{\tiny Photo: Ralf Rebmann, CC BY-SA 4.0}


\end{frame}



% the fellowship project
\begin{frame}{Problems to solve}
  %

  \vspace{-0.5cm}
  
  \begin{columns}
    %
    \begin{column}{.6\textwidth}
      \minipage[c][0.2\textheight][s]{\columnwidth}

      \vfill
      
      Problem 1 

      \begin{itemize}[leftmargin=0.6cm]
        
      \item[-] using of difficult to install computational environment (graph-tool)       
        
      \end{itemize}

      \vfill
      
      
      \endminipage      
    \end{column}
    %
    \begin{column}{.4\textwidth}



      \begin{figure}
        \centering
        \includegraphics<2->[width=0.75\textwidth]{%
          img/docker_logo.pdf} %
      \end{figure}
      
      
    \end{column}
  \end{columns}


  \vspace{-0.3cm}
  \begin{columns}
    %
    \begin{column}{.6\textwidth}
      \minipage[c][0.5\textheight][s]{\columnwidth}

      
      \vfill
      \textcolor{white}{
      Problem 2        }

      
      \endminipage      
    \end{column}
    %
    \begin{column}{.4\textwidth}
      %% \vspace{0.1cm}
      %% \begin{figure}
      %%   \centering
      %%   \includegraphics<6->[width=0.85\textwidth]{%
      %%     img/sumatra_logo.png} %
      %% \end{figure}

      
      
      
    \end{column}
  \end{columns}

  %% \source{\onslide<6>{\cite{Davison2012}}}
\end{frame}

\include{frames/prototype_docker}
\begin{frame}{Problems to solve}
  %

  \vspace{-0.5cm}
  
  \begin{columns}
    %
    \begin{column}{.6\textwidth}
      \minipage[c][0.2\textheight][s]{\columnwidth}

      \vfill
      
      Problem 1 \onslide<2->{Check}

      \begin{itemize}[leftmargin=0.6cm]
        
      \item[-] using of difficult to install computational environment (graph-tool)       
        
      \end{itemize}

      \vfill
      
      
      \endminipage      
    \end{column}
    %
    \begin{column}{.4\textwidth}



      \begin{figure}
        \centering
        \includegraphics[width=0.75\textwidth]{%
          img/docker_logo.pdf} %
      \end{figure}
      



      
    \end{column}
  \end{columns}


  \onslide<3->
  \vspace{-0.3cm}
  \begin{columns}
    %
    \begin{column}{.6\textwidth}
      \minipage[c][0.5\textheight][s]{\columnwidth}

      \vfill
      
      Problem 2        

      \begin{itemize}[leftmargin=0.6cm]
        
      \item<3->[-] long \& resource demanding computations
      \item<4->[-] subsequent analysis require output of previous computations \vspace{0.17cm}
      \item<5->[] \textit{difficult to understand what is required to reproduce a single output (1 figure)}
        
      \end{itemize}


      
      \endminipage      
    \end{column}
    %
    \begin{column}{.4\textwidth}
      \vspace{0.1cm}
      \begin{figure}
        \centering
        \includegraphics<6->[width=0.85\textwidth]{%
          img/sumatra_logo.png} %
      \end{figure}

      
      
      
    \end{column}
  \end{columns}  
\end{frame}

\begin{frame}{}
  
  \begin{center}
    \inlineMovie[autostart&stop=50]{movie/smtweb.mp4}{movie/smtweb.png}{height=0.9\textheight}

  \end{center}
  
\end{frame}





\begin{frame}{Problems to solve}
  %

  \vspace{-0.5cm}
  
  \begin{columns}
    %
    \begin{column}{.6\textwidth}
      \minipage[c][0.2\textheight][s]{\columnwidth}

      \vfill
      
      Problem 1 \onslide<1->{\textcolor{ForestGreen}{\LARGE $\,\,$\cmark}}

      \begin{itemize}[leftmargin=0.6cm]
        
      \item[-] using of difficult to install computational environment (graph-tool)       
        
      \end{itemize}

      \vfill
      
      
      \endminipage      
    \end{column}
    %
    \begin{column}{.4\textwidth}



      \begin{figure}
        \centering
        \includegraphics[width=0.75\textwidth]{%
          img/docker_logo.pdf} %
      \end{figure}
     
    \end{column}
  \end{columns}


  \onslide<1->
  \vspace{-0.3cm}
  \begin{columns}
    %
    \begin{column}{.6\textwidth}
      \minipage[c][0.5\textheight][s]{\columnwidth}

      \vfill
      
      Problem 2 \onslide<2->{\textcolor{ForestGreen}{\LARGE $\,\,$\cmark}}

      \begin{itemize}[leftmargin=0.6cm]
        
      \item<1->[-] long \& resource demanding computations
      \item<1->[-] subsequent analysis require output of previous computations \vspace{0.17cm}
      \item<1->[] \textit{difficult to understand what is required to reproduce a single output (1 figure)}
        
      \end{itemize}


      
      \endminipage      
    \end{column}
    %
    \begin{column}{.4\textwidth}
      \vspace{0.1cm}
      \begin{figure}
        \centering
        \includegraphics<1->[width=0.85\textwidth]{%
          img/sumatra_logo.png} %
      \end{figure}

      
      
      
    \end{column}
  \end{columns}

  \source{\onslide<1->{\cite{Davison2012}}}
\end{frame}

\begin{frame}{}
  
  \begin{center}
    \inlineMovie[autostart]{movie/smt_repeat.mp4}{movie/smt_repeat.png}{height=0.9\textheight}

  \end{center}
  
\end{frame}





\begin{frame}{Computational reproducibility in a prototype}

  \begin{figure}
    \centering
    \includegraphics[width=0.3\textwidth]{%
      docker_logo} %
    \hspace{1cm}  {\Huge +} \hspace{1cm}
    \includegraphics[width=0.3\textwidth]{%
      sumatra_logo} %
  \end{figure}

  \vspace{-0.2cm}
  \begin{center}
    {\Huge = }
  \end{center}
  \vspace{-0.35cm}

  \begin{center}
    {\Large computational reproducibility (in a prototype)}
  \end{center}

  \vspace{0.4cm}
  
    \begin{columns}
      %
      \begin{column}{.5\textwidth}
        \large
        \begin{center}
          Example study:\\
          \href{http://bit.ly/osproj}{http://bit.ly/osproj}
        \end{center}

      \end{column}
      %
      \begin{column}{.5\textwidth}
        \begin{center}
          \large Documentation:\\
          \href{http://bit.ly/osrep}{http://bit.ly/osrep}
        \end{center}
        
      \end{column}
    \end{columns}
   

\end{frame}




% So what can I take out of this talk? Give some rules
% and guide lines to follow (other resources)
% -> Benureau & Rougier's 5Rs


\begin{frame}{Five $R$s for reproducible scientific code}

  \small
  
  \begin{itemize}[leftmargin=2.45cm]
    \itemsep9pt
    \item<1->[$\mathbf{R^1}$ \textit{Re-runnable}:] can be run again when needed \onslide<2->{\textcolor{mpipurp}{$\rightarrow$ document dependencies necessary to run code}}

    \item<3->[$\mathbf{R^2}$ \textit{Repeatable}:] program is deterministic, produces repeatable output \onslide<4->{\textcolor{mpipurp}{$\rightarrow$ add seeds for random number generators}}

    \item<5->[$\mathbf{R^3}$ \textit{Reproducible}:] another researcher can take code \& input data, execute code, and re-obtain same results \\ \onslide<6->{\textcolor{mpipurp}{$\rightarrow$ detailed versions of dependencies, version of code, \ul{availability}}}

          \item<1->[] \textcolor{white}{program  can be easily used, and modified, by you and other people, inside \& outside own lab \\ \onslide<1->{$\rightarrow$ avoid hard coded numbers, write documentation}}

    \item<1->[] \textcolor{white}{program can be re-implemented by another research to re-obtain results  \\ \onslide<1->{$\rightarrow$ see for example ReScience journal}}



  \end{itemize}  

  
  \source{\cite{Benureau2018}}
  
  
\end{frame}


\begin{frame}{Where to publish code}

  \only<1>{\vspace{0.4cm}}

  \begin{figure}
    \centering
    \includegraphics<1>[width=0.7\textwidth]{%
      zenodo.png} %
    \includegraphics<2>[width=0.95\textwidth]{%
      osf_example.png} %
  \end{figure}

  \begin{center}
    \only<1>{\href{https://zenodo.org/}{https://zenodo.org/}}    
  \end{center}

  \begin{center}
    \only<2>{\vspace{-0.8cm}\href{https://osf.io/}{https://osf.io/}}
  \end{center}


  
  
  
\end{frame}





\begin{frame}{Five $R$s for reproducible scientific code}

  \small
  
  \begin{itemize}[leftmargin=2.45cm]
    \itemsep9pt
    \item<1->[$\mathbf{R^1}$ \textit{Re-runnable}:] can be run again when needed \onslide<1->{\textcolor{mpipurp}{$\rightarrow$ document dependencies necessary to run code}}

    \item<1->[$\mathbf{R^2}$ \textit{Repeatable}:] program is deterministic, produces repeatable output \onslide<1->{\textcolor{mpipurp}{$\rightarrow$ add seeds for random number generators}}

    \item<1->[$\mathbf{R^3}$ \textit{Reproducible}:] another researcher can take code \& input data, execute code, and re-obtain same results \\ \onslide<1->{\textcolor{mpipurp}{$\rightarrow$ detailed versions of dependencies, version of code, \ul{availability}}}

    \item<2->[$\mathbf{R^4}$ \textit{Reusable}:] program  can be easily used, and modified, by you and other people, inside \& outside own lab \\ \onslide<3->{\textcolor{mpipurp}{$\rightarrow$ avoid hard coded numbers, write documentation}}

    \item<4->[$\mathbf{R^5}$ \textit{Replicable}:] program can be re-implemented by another research to re-obtain results  \\ \onslide<5->{\textcolor{mpipurp}{$\rightarrow$ see for example ReScience journal}}

  \end{itemize}  

  
  \source{\cite{Benureau2018}}
  
  
\end{frame}


% cite these nice papers
\nocite{Collberg2013, Rougier2017}
\include{frames/bibliography}

% overflow


\begin{frame}{$R^0$ -- runnable code}

  \begin{figure}
    \centering
    \includegraphics[width=\textwidth]{%
      img/R0_code.png} %
  \end{figure}

  \begin{figure}
    \centering
    \includegraphics[width=\textwidth]{%
      img/R0_output.png} %
  \end{figure}

  \vspace{1.5cm}  

  \source{\cite{Benureau2018}}
  
  
\end{frame}



\begin{frame}{$R^1$ -- Re-runnable}

  \begin{figure}
    \centering
    \includegraphics[width=\textwidth]{%
      img/R1_code.png} %
  \end{figure} 
  
  
\end{frame}



\begin{frame}{$R^2$ -- Repeatable}
  
    \begin{figure}
    \centering
    \includegraphics[width=\textwidth]{%
      img/R2_code.png} %
    \end{figure} 
      
\end{frame}



\begin{frame}{$R^3$ -- Reproducible}

  \begin{figure}
    \centering
    \includegraphics<1>[width=.8\textwidth]{%
      img/R3_code01.png} %
    \includegraphics<2>[width=.8\textwidth]{%
      img/R3_code02.png} %   
  \end{figure}

  \source{\only<1>{1}\only<2>{2}/2}
    
\end{frame}


\begin{frame}{$R^4$ -- Reusable}

  \begin{figure}
    \centering
    \includegraphics<1>[width=.8\textwidth]{%
      img/R4_code01.png} %
    \includegraphics<2>[width=.8\textwidth]{%
      img/R4_code02.png} %
    \includegraphics<3>[width=.8\textwidth]{%
      img/R4_code03.png} %   
  \end{figure}

    \source{\only<1>{1}\only<2>{2}\only<3>{3}/3}
    
\end{frame}



\begin{frame}{$R^5$ -- Replicable}

  \begin{figure}
    \centering
    \includegraphics<1>[width=.8\textwidth]{%
      img/R5_code01.png} %
    \includegraphics<2>[width=.8\textwidth]{%
      img/R5_code02.png} %   
  \end{figure}

  \source{\only<1>{1}\only<2>{2}/2}
    
\end{frame}


\end{document}
















