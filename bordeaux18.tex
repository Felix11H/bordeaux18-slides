
\documentclass[11pt]{beamer}

% contains the complete preamble
\usepackage{present}
\usepackage{custom_underline}
         		               		
\title{}
\author[Felix Z.~Hoffmann]{Felix Z.~Hoffmann}
\institute{}
\date{\today}     


\begin{document}


\begin{frame}
  %
  \vspace{0.6cm}
  
  \begin{center} 
    \Large Open \& reproducible research - What can we do in practice?
  \end{center}
  
  \vspace{0.8cm}

  \small
  \begin{columns}[t]
    %%%%
    \begin{column}{0.4\textwidth}        
      \textbf{Presented by}
      
      %\vspace{0.18cm}
      \begin{itemize}[leftmargin=0.6cm]
        \itemsep0pt
      \item[] Felix Z.~Hoffmann
        \item[] @Felix11H
        \item[] \href{http://felix11h.github.io/}{felix11h.github.io/}
      \end{itemize}

      \vspace{0.28cm}
      \textbf{Slides}
      
      %\vspace{0.18cm}
      \begin{itemize}[leftmargin=0.6cm]
        %% \item[] Slideshare:\\ %
        %%   \href{}{git.io/vFEbX}
        \item[] GitHub: %
          \href{http://bit.ly/bx18s}{bit.ly/bx18s}
      \end{itemize}
    \end{column}
    %%%% 
    \begin{column}{0.59\textwidth}
      \textbf{Resources and links}
      \vspace{-0.1cm}
       
      \begin{itemize}[leftmargin=0.6cm]
        \itemsep4pt
        \item[] Open Science Fellows Program: %
          \href{http://bit.ly/osfprog}{bit.ly/osfprog}
        \item[] project description: %
          \href{http://bit.ly/osfproj}{bit.ly/osfproj}
        \item[] prototype: \href{http://bit.ly/osrep}{bit.ly/osrep}
      \end{itemize}



    \end{column}
    %%%%
  \end{columns}

  \vspace{0.5cm}

  %% \begin{center}
  %%   \includegraphics[width=1.7cm]{ccby40.png}

  %%   \vspace{0.1cm}

  %%   \scriptsize This work is licensed under a
  %%   \href{https://creativecommons.org/licenses/by/4.0/}{Creative
  %%     Commons Attribution 4.0 International License.}
  %% \end{center}

  %% \begin{center}
  %%   \includegraphics[width=\textwidth]{img/gsoc14_banner.png}
  %% \end{center}

  \begin{figure}
  \centering
  \includegraphics[width=\textwidth]{%
  img/wosf_header.jpg} %
\end{figure}



\end{frame}



% pre-Intro: About me - what qualifies me to tell you
% anything about reproducibility/open science. Whom am
% I to talk?

\begin{frame}{}

  - Research Lab

  - Google Summer of Code

  - Wikimedia Open Science Fellow

   
  
\end{frame}



% Intro: Why we (might) need to do things differently
% from our supervisors/how it's been done until now
% reproducibility crisis everywhere !!
\begin{frame}{The reproducibility crisis}

  \begin{figure}
    \centering
    \includegraphics[width=0.85\textwidth]{%
    img/reproducibility_crisis_q.jpeg} %
  \end{figure}
  

  
  
\end{frame}



% So let's talk about code. Probably all of us are working
% with code, even if it's just a simple MATLAB script.
%
% What does reproducibility for these computational parts
% look like?
% 
% Naively, reproducibility shouldn't be a problem at all for code,
% right?
%
% 1. everyone has a computer (as opposed to a lab setup) and
%    running code is inexpensive and unproblematic compared
%    to replicating experiment
% 2. can share all the code (~5MB) opposed to sharing a full
%    lab (impossible)
% 
% So where's reproducibility in the computational sciences at?
% Well, it's not looking great.
%
% Many of you probably know from personal experience that this
% that this naive thinking does not apply, but let's look at
% how bad it really is:
%
% Problems:
% ---> study from Stodden and Collberg

\begin{frame}{Computational reproducibility should be easy...}

  % 
  \begin{columns}
    %
    \begin{column}{.5\textwidth}
      \begin{figure}
        \centering
        \includegraphics[width=0.85\textwidth]{%
          monkey.png}  %
      \end{figure}
      
    \end{column}
    %
    \begin{column}{.5\textwidth}

      \begin{figure}
        \centering
        \includegraphics[width=0.85\textwidth]{%
          computer.png}  %
      \end{figure}
      
    \end{column}
  \end{columns}

  \begin{columns}
    %
    \begin{column}{.5\textwidth}
      \begin{center}
        Hard
      \end{center}
      
    \end{column}
    %
    \begin{column}{.5\textwidth}
      \begin{center}
        Easy?
      \end{center}
      
    \end{column}
  \end{columns}
  
\end{frame}


\begin{frame}{Computational reproducibility should be easy...}

  \begin{itemize}[leftmargin=1cm]

  \item[1.] everyone has access to computers (as opposed to a lab setup)
  \item[2.] running code is inexpensive and unproblematic (compared to replicating an experiment)
  \item[3.] can easily share code \& data 
    % 1. everyone has a computer (as opposed to a lab setup) and
    %    running code is inexpensive and unproblematic compared
    %    to replicating experiment
    % 2. can share all the code (~5MB) opposed to sharing a full
    %    lab (impossible)
    
    
  \end{itemize}

  \begin{center}
    $\Rightarrow$ Do we need to care?
  \end{center}

  
\end{frame}



\begin{frame}{Sharing of code \& data mandatory in many journals}

  
  \begin{figure}
    \centering
    \includegraphics<1>[width=0.95\textwidth]{%
      img/science_policy.png} %
    \includegraphics<2>[width=0.95\textwidth]{%
      img/science_policy_t1.png} %
    \includegraphics<3>[width=0.95\textwidth]{%
      img/science_policy_t2.png} %
  \end{figure}

  \vspace{0.4cm}
  
  \begin{flushright}
    \small Policy of \textit{Science} since February 11, 2011
  \end{flushright}

  \source{\cite{Stodden2018}}

  \pnote{
    
    Now that we established computational reproducibility\\
    should be simple and journals are demanding\\
    sharing anyway, we're good, right?
    
  }
  
\end{frame}

\begin{frame}{}
  
  \begin{figure}
    \centering
    \includegraphics[width=0.875\textwidth]{%
    img/stodden2018_cover3.png} %
  \end{figure}
  
  \vspace{0.015cm}

  \begin{center}
  Out of 206 computational studies in \textit{Science} since 2011, 26 provided code \& data directly  
  \end{center}
  

  \source{\cite{Stodden2018}}
  

  
\end{frame}


\begin{frame}{A few responses...}
  
  \begin{figure}
    \centering
    \includegraphics[width=0.95\textwidth]{%
      img/stodden2018_re1.png} %
  \end{figure}

  \begin{figure}
    \centering
    \includegraphics[width=0.95\textwidth]{%
      img/stodden2018_re2.png} %
  \end{figure}

  \begin{figure}
    \centering
    \includegraphics[width=0.95\textwidth]{%
      img/stodden2018_re3.png} %
  \end{figure}

  \source{\cite{Stodden2018}}
  
\end{frame}

\begin{frame}{But also...}
  
  \begin{figure}
    \centering
    \includegraphics[width=0.95\textwidth]{%
      img/stodden2018_re4.png} %
  \end{figure}

  \begin{figure}
    \centering
    \includegraphics[width=0.95\textwidth]{%
      img/stodden2018_re5.png} %
  \end{figure}

  \begin{figure}
    \centering
    \includegraphics[width=0.95\textwidth]{%
      img/stodden2018_re6.png} %
  \end{figure}

  \source{\cite{Stodden2018}}
  
\end{frame}


\begin{frame}{\large   Of $N=206$ articles published in \textit{Science} since 2011...}


  \begin{figure}
    \centering
    \includegraphics[width=0.9\textwidth]{%
      img/stodden2018_table1.png} %
  \end{figure}

  \vspace{0.1cm}
  
  \begin{center}
    $\Rightarrow$ Code \& data could be retrieved for 91 out of 206 studies.    
  \end{center}

  \source{\cite{Stodden2018}}
  
\end{frame}


\begin{frame}{}%Open Source for Neuroscience}

  \begin{figure}
    \includegraphics[width=0.95\textwidth]{%
      img/open_source_neuroscience.png} %
  \end{figure}
  
\end{frame}



\begin{frame}{Open Source for Neuroscience}

  \begin{figure}
    \centering
    %% \includegraphics<1>[width=0.9\textwidth]{%
    %%   img/open_source_neuroscience.png} %
    \includegraphics<1>[width=\textwidth]{%
      img/pledge_2.png} %
    \includegraphics<2>[width=\textwidth]{%
      img/pledge_2_t1.png} %
  \end{figure}

  \onslide<1->
  \begin{center}
    \href{http://opensourceforneuroscience.org/}{opensourceforneuroscience.org/}
  \end{center}

  \pnote{
    
    As of 2018/04: 200 neuroscientist have made this pledge
    
  }
  
\end{frame}



\begin{frame}{Reproducibility when code was available}

  \vspace{0.05cm}
  
  56 out of 91 studies were judged as \textit{potentially reproducible}

  \vfill
  
  Testing 22 randomly selected studies:

  \vspace{0.1cm}

  \only<1-3>{
    \begin{figure}
      \centering
      \includegraphics<1>[width=0.825\textwidth]{%
        img/stodden2018_table4_1-3.png} %
      \includegraphics<2>[width=0.825\textwidth]{%
        img/stodden2018_table4_2-3.png} %
      \includegraphics<3>[width=0.825\textwidth]{%
        img/stodden2018_table4_3-3.png} %
  \end{figure}}

  \only<4>{
    \begin{center}
      \begin{tikzpicture}[every text node part/.style={align=center}]
        \draw (0, 0) node[inner sep=0] {\includegraphics<4>[width=0.825\textwidth]{%
            img/stodden2018_table4_3-3.png}};
        \draw  (0, -2.25) node {Even when code was available, more than half of studies\\ were reproducible only with \textit{\textcolor{red}{significant effort!}}};
      \end{tikzpicture}
      
    \end{center}
  }

  \source{\cite{Stodden2018}}
  
  
  
  
\end{frame}





\begin{frame}{\large Measuring Reproducibility in Computer Systems Research}

  \begin{figure}
    \centering
    \includegraphics[width=\textwidth]{%
      img/collberg2013_results3.png} %
  \end{figure}
  

  \source{\cite{Collberg2013}}

  \pnote{
    
    613 papers from \\
    - 8 conferences \\
    - 5 journals

    30 minutes of programmer time to try \\
    make build compile/run
    
    Orange: Don't send more than one email to any author \\
    (if author had multiple publications)

    Even if build runs does not even try verify results! \\
    How many more papers will fall off?
    
  }
  
  
\end{frame}


\begin{frame}{\large Why wasn't code available? (Or, the dog ate my program)}
  
  
  
\end{frame}




% So what's there to do?
% ---> Be open, be reproducible (in that order)
% many people are already doing it?
% ---> open source for Neuroscience (idea: highlight
% researchers from Bordeaux/Frankfurt?)
\begin{frame}{Open Source for Neuroscience}

  \begin{figure}
    \centering
    \includegraphics<1>[width=0.9\textwidth]{%
      img/open_source_neuroscience.png} %
    \includegraphics<2>[width=\textwidth]{%
      img/pledge_2.png} %
    \includegraphics<3>[width=\textwidth]{%
      img/pledge_underline_2.png} %
  \end{figure}

  \onslide<2->
  \begin{center}
    \href{http://opensourceforneuroscience.org/}{opensourceforneuroscience.org/}
  \end{center}

  \pnote{
    
    As of 2018/04: 200 neuroscientist have made this pledge
    
  }
  
\end{frame}



% Next, we need reproducibility! What is even
% reproducible? Does open==reproducible?
% So what does code need to look like in practice?
% --> Rougier's five Rs


\begin{frame}{Five $R$s for reproducible scientific code}

  \small
  
  \begin{itemize}[leftmargin=2.45cm]
    \itemsep9pt
    \item<1->[$\mathbf{R^1}$ \textit{Re-runnable}:] can be run again when needed \onslide<2->{\textcolor{mpipurp}{$\rightarrow$ document dependencies necessary to run code}}

    \item<3->[$\mathbf{R^2}$ \textit{Repeatable}:] program is deterministic, produces repeatable output \onslide<4->{\textcolor{mpipurp}{$\rightarrow$ add seeds for random number generators}}

    \item<5->[$\mathbf{R^3}$ \textit{Reproducible}:] another researcher can take code \& input data, execute code, and re-obtain same results \\ \onslide<6->{\textcolor{mpipurp}{$\rightarrow$ detailed versions of dependencies, version of code, \ul{availability}}}

          \item<1->[] \textcolor{white}{program  can be easily used, and modified, by you and other people, inside \& outside own lab \\ \onslide<1->{$\rightarrow$ avoid hard coded numbers, write documentation}}

    \item<1->[] \textcolor{white}{program can be re-implemented by another research to re-obtain results  \\ \onslide<1->{$\rightarrow$ see for example ReScience journal}}



  \end{itemize}  

  
  \source{\cite{Benureau2018}}
  
  
\end{frame}


\begin{frame}{Where to publish code}

  \only<1>{\vspace{0.4cm}}

  \begin{figure}
    \centering
    \includegraphics<1>[width=0.7\textwidth]{%
      zenodo.png} %
    \includegraphics<2>[width=0.95\textwidth]{%
      osf_example.png} %
  \end{figure}

  \begin{center}
    \only<1>{\href{https://zenodo.org/}{https://zenodo.org/}}    
  \end{center}

  \begin{center}
    \only<2>{\vspace{-0.8cm}\href{https://osf.io/}{https://osf.io/}}
  \end{center}


  
  
  
\end{frame}





\begin{frame}{Five $R$s for reproducible scientific code}

  \small
  
  \begin{itemize}[leftmargin=2.45cm]
    \itemsep9pt
    \item<1->[$\mathbf{R^1}$ \textit{Re-runnable}:] can be run again when needed \onslide<1->{\textcolor{mpipurp}{$\rightarrow$ document dependencies necessary to run code}}

    \item<1->[$\mathbf{R^2}$ \textit{Repeatable}:] program is deterministic, produces repeatable output \onslide<1->{\textcolor{mpipurp}{$\rightarrow$ add seeds for random number generators}}

    \item<1->[$\mathbf{R^3}$ \textit{Reproducible}:] another researcher can take code \& input data, execute code, and re-obtain same results \\ \onslide<1->{\textcolor{mpipurp}{$\rightarrow$ detailed versions of dependencies, version of code, \ul{availability}}}

    \item<2->[$\mathbf{R^4}$ \textit{Reusable}:] program  can be easily used, and modified, by you and other people, inside \& outside own lab \\ \onslide<3->{\textcolor{mpipurp}{$\rightarrow$ avoid hard coded numbers, write documentation}}

    \item<4->[$\mathbf{R^5}$ \textit{Replicable}:] program can be re-implemented by another research to re-obtain results  \\ \onslide<5->{\textcolor{mpipurp}{$\rightarrow$ see for example ReScience journal}}

  \end{itemize}  

  
  \source{\cite{Benureau2018}}
  
  
\end{frame}



% Is this at all possible to implement?  Does it
% really work? Yes! I think. Here's an example
% how things can be done --> My project
\begin{frame}{Wikimedia Open Science Fellowships}

  - picture of the current fellows
  - application for 3rd round now open
  
\end{frame}



% A few tips in practice:

% -- [1] upload to Zenodo, OSF.io, ---> DOI
% -- [2] use random seeds and makefile to regenerate results
% -- [3] document versions of code (toolboxes), (pip freeze, MATLAB versions)
% -- [4] provide dependencies (Docker, Singularity, Virtual Machine,...)
% -- [5] document more closely parameters, input/output, ...
\include{frames/in_practice}



% So what can I take out of this talk? Give some rules
% and guide lines to follow (other resources)?

\begin{frame}{}

  - Zenodo
  - Open Science Framework
  -  
  
\end{frame}



\include{frames/bibliography}



\end{document}
















