
\documentclass[9pt]{beamer}

% contains the complete preamble
\usepackage{present}
\usepackage{custom_underline}


\definecolor{preorange}{RGB}{229,129,79}
\definecolor{pregreen}{RGB}{196,202,52}
	         		               		
\title{}
\author[Felix Z.~Hoffmann]{Felix Z.~Hoffmann}
\institute{}
\date{\today}     


\begin{document}


\begin{frame}
  %
  \vspace{0.6cm}
  
  \begin{center} 
    \Large Open \& reproducible research - What can we do in practice?
  \end{center}
  
  \vspace{0.8cm}

  \small
  \begin{columns}[t]
    %%%%
    \begin{column}{0.4\textwidth}        
      \textbf{Presented by}
      
      %\vspace{0.18cm}
      \begin{itemize}[leftmargin=0.6cm]
        \itemsep0pt
      \item[] Felix Z.~Hoffmann
        \item[] @Felix11H
        \item[] \href{http://felix11h.github.io/}{felix11h.github.io/}
      \end{itemize}

      \vspace{0.28cm}
      \textbf{Slides}
      
      %\vspace{0.18cm}
      \begin{itemize}[leftmargin=0.6cm]
        %% \item[] Slideshare:\\ %
        %%   \href{}{git.io/vFEbX}
        \item[] GitHub: %
          \href{http://bit.ly/bx18s}{bit.ly/bx18s}
      \end{itemize}
    \end{column}
    %%%% 
    \begin{column}{0.59\textwidth}
      \textbf{Resources and links}
      \vspace{-0.1cm}
       
      \begin{itemize}[leftmargin=0.6cm]
        \itemsep4pt
        \item[] Open Science Fellows Program: %
          \href{http://bit.ly/osfprog}{bit.ly/osfprog}
        \item[] project description: %
          \href{http://bit.ly/osfproj}{bit.ly/osfproj}
        \item[] prototype: \href{http://bit.ly/osrep}{bit.ly/osrep}
      \end{itemize}



    \end{column}
    %%%%
  \end{columns}

  \vspace{0.5cm}

  %% \begin{center}
  %%   \includegraphics[width=1.7cm]{ccby40.png}

  %%   \vspace{0.1cm}

  %%   \scriptsize This work is licensed under a
  %%   \href{https://creativecommons.org/licenses/by/4.0/}{Creative
  %%     Commons Attribution 4.0 International License.}
  %% \end{center}

  %% \begin{center}
  %%   \includegraphics[width=\textwidth]{img/gsoc14_banner.png}
  %% \end{center}

  \begin{figure}
  \centering
  \includegraphics[width=\textwidth]{%
  img/wosf_header.jpg} %
\end{figure}



\end{frame}



% pre-Intro: About me - what qualifies me to tell you
% anything about reproducibility/open science. Whom am
% I to talk?

\begin{frame}{}

  - Research Lab

  - Google Summer of Code

  - Wikimedia Open Science Fellow

   
  
\end{frame}


% Intro: Why we (might) need to do things differently
% from our supervisors/how it's been done until now
\begin{frame}{The reproducibility crisis}

  \begin{figure}
    \centering
    \includegraphics[width=0.85\textwidth]{%
    img/reproducibility_crisis_q.jpeg} %
  \end{figure}
  

  
  
\end{frame}


% So what can we do? -> A call for open source in
% neuroscience
\begin{frame}{Open Source for Neuroscience}

  \begin{figure}
    \centering
    \includegraphics<1>[width=0.9\textwidth]{%
      img/open_source_neuroscience.png} %
    \includegraphics<2>[width=\textwidth]{%
      img/pledge_2.png} %
    \includegraphics<3>[width=\textwidth]{%
      img/pledge_underline_2.png} %
  \end{figure}

  \onslide<2->
  \begin{center}
    \href{http://opensourceforneuroscience.org/}{opensourceforneuroscience.org/}
  \end{center}

  \pnote{
    
    As of 2018/04: 200 neuroscientist have made this pledge
    
  }
  
\end{frame}


% But what does this mean? Does open == reproducible?
% What do we need to do to make our code reproducible?
% --> Rougier's five Rs
\include{frames/what_open_and_reproducible_means}

% Does it really work? Yes! I think. Here's an example
% how things can be done --> My project
\begin{frame}{Wikimedia Open Science Fellowships}

  - picture of the current fellows
  - application for 3rd round now open
  
\end{frame}



% So what can I take out of this talk? Give some rules
% and guide lines to follow (other resources)?

\begin{frame}{}

  - Zenodo
  - Open Science Framework
  -  
  
\end{frame}



\include{frames/bibliography}



\end{document}
















